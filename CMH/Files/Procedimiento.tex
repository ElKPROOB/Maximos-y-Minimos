\documentclass[]{article}

\usepackage{amsmath}
\usepackage{amssymb}
\usepackage[utf8]{inputenc}
\usepackage[spanish]{babel}

\begin{document}{\large Función: }$x^{2} + y^{2}$\\

\noindent
- Puntos singulares:\\

\noindent
\hspace{1cm}$\Rightarrow$ La función es diferenciable en su dominio\\

\noindent
- La función no tiene puntos frontera\\

\noindent
- Puntos estacionarios:\\

\[
\text{\textit{Vector Gradiente: }}
\]
\[
\left(
\begin{array}{ccc}
2 x, &
2 y
\end{array}
\right)
\]\\

\noindent
\hspace{1cm}$\rightarrow$ Puntos en los que el vector gradiente es 0:\\

\textbf{\hspace{1cm}$\bullet$ $(x: 0,\hspace{2mm}y: 0)$\\}

\[
\text{\textit{Matriz Hessiana: }}
\]
\[
\left(
\begin{array}{ccc}
2 & 0\\
0 & 2\\
\end{array}
\right)
\]\\

\[
\text{\textit{Matriz Hessiana Evaluada en el punto $(x: 0,\hspace{2mm}y: 0)$: }}
\]
\[
\left(
\begin{array}{ccc}
2 & 0\\
0 & 2\\
\end{array}
\right)
\]\\

\[
\Rightarrow\left(
\begin{array}{ccc}
2 - \lambda & 0\\
0 & 2 - \lambda\\
\end{array}
\right)
\]\\

\[
\Rightarrow \left(2 - \lambda\right)^{2} = 0
\]\\

\[
\Rightarrow \text{Los posibles valores para $\lambda$ son: }\lambda=2
\]\\

\textbf{\hspace{1cm}$\therefore$ Se tiene un \underline{Mínimo Relativo} en el punto $(x: 0,\hspace{2mm}y: 0)$\\}

\end{document}